\documentclass[fontsize=55pt,  paper=a2, pagesize]{scrartcl}
% page number suppressed 
\pagenumbering{gobble}
% left aligned text
\usepackage[document]{ragged2e}
% load images
\usepackage{graphicx}
% to crop images more easily (Export overrides \includegraphics)
%\usepackage[Export]{adjustbox}

\usepackage[cmyk]{xcolor}
\usepackage{setspace}

\definecolor{pepgray}{HTML}{565656}
\colorlet{strokeColor}{white}
\colorlet{fillColor}{pepgray}
\colorlet{textColor}{black}

% set font almost like poster
\usepackage{fontspec}
\setmainfont{Roboto}
\setsansfont{Roboto}
\renewcommand{\familydefault}{\sfdefault}

% get control over borders
\usepackage[ a2paper, margin=1cm, %
            ignoreall, noheadfoot, nomarginpar, %top=-1in %
            %showframe %
            ]{geometry}
% use tikz
\usepackage{tikz}
\usetikzlibrary{calc}
\usetikzlibrary{svg.path}

% Hyperlinks
\usepackage[
  allbordercolors = pepgray,
  unicode,        % Unicode in PDF-Attributen erlauben
  pdfusetitle,    % Titel, Autoren und Datum als PDF-Attribute
  pdfcreator={},  % ┐ PDF-Attribute säubern
  pdfproducer={}, % ┘
]{hyperref}

\begin{document}
\thispagestyle{empty}

% set new length unit
\newlength\ulen
\setlength\ulen{1cm}
\setlength\parindent{0cm}


% use tikzpicture environment to move objects around 
% options: use tikz as overlay, shift that (0,0) os upper left corner and set the default length units
\begin{tikzpicture}[remember picture, overlay, shift=(current page.north west), x=\ulen, y=\ulen]
  % test units and borders: 
  %\draw[->] (0,0) -- (2,-2);
  %\draw[line width=0.1cm] (current page.north west) rectangle (current page.south east);
  % Logo jdpg

  \begin{scope}[shift={($(current page.north)+(-9, -8)$)}]
    \node[anchor=west,text width=30\ulen, color=textColor] (pep) at (0, 0) {%
    {\huge\bfseries PeP et al.~e.\kern-0.1333em V\kern-0.1333em.}\\[1ex]
    {\setstretch{0.7}%
    Physikstudierende und\\
    ehemalige Physikstudierende\\
    der TU Dortmund\par
    }
    };
    \begin{scope}[anchor=east, scale=5, shift={($(pep.west)+(-1, 0)$)}]
      \fill[color=fillColor] (0, 0) circle[radius=1cm];
      \draw[
        color=strokeColor,
        line width=1pt,
        smooth,
        yscale=0.9,
        xscale=0.57,
        xshift=-0.5cm,
        domain=-1.20:2.25,
        samples=200,
      ] plot (\x, {exp(-\x * \x) * cos(10 * deg(\x))});
    \end{scope}
  \end{scope}

  \node[text width=30\ulen, anchor=north, color=textColor, align=justify] at ($(current page.north)+(0, -16)$)
  {Wir sind der ehemaligenverein und können uns keinen Demotext leisten! Waas! Keinen Demotext?
    Kostet ein Demotext überhaupt etwas? Eigentlich nicht. Aber ich muss ja auch mal wieder in langen,
    inhaltlich unanpspruchsvollen texten mein 10-finger-tippen üben! Wie wäre es jetzt eigentlich mit einer liste?
    \begin{itemize}
      \item Toller Punkt
      \item kekse!
    \end{itemize}
  };

  \begin{scope}[shift={($(current page.west)+(39.5, -15)$)}, xscale=-9, yscale=9]
    % \fill[color=fillColor] (0, 0) circle[radius=1\ulen];
    \fill[
      color=pepgray,
      line width=1pt,
      smooth,
      yscale=0.9,
      xscale=0.57,
      xshift=-0.5cm,
      domain=-1.20:2.25,
      samples=200,
    ] plot (\x, {exp(-\x * \x) * cos(10 * deg(\x))}) -- ++(20, 0) -- ++(0, -20) -- ++(-40, 0) -- ++(0, 20)  -- cycle;
  \end{scope}
  \node[anchor=south, color=white, text width=\linewidth, align=center] (link) at ($(current page.south)+(0, 5)$)
  {\footnotesize Alle Infos kriegs'te hier: \\[1ex] \href{https://pep-dortmund.de/mitmachen.html}{https://pep-dortmund.de/mitmachen.html}};
\end{tikzpicture}

\end{document}
